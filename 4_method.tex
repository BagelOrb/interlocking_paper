

\section{Method}


\newpage

\subsection{Analytically optimal structure}

When any structure would fail by breakage or plastic deformation,
the structure could be enhanced by reinforcing the location where that fault would happen.
Our interlocking structure consists of beams of two materials.
Reinforcing the beams of one material means that we reduce beams of the other material.
If we consider beams of a homogeneous width then the respective widths of the beams is subject to optimization.
The width ratio is optimal when both materials would fail under the same load:

\begin{align*}
	F_\text{PLA} &= F_\text{PP} \\
	A_\text{PLA} \times \sigma_\text{PLA} &= 	A_\text{PP} \times \sigma_\text{PP}\\
	\sigma_\text{PLA} &\approx 37 \si{\mega\pascal} \\
	\sigma_\text{PP} &\approx 8.7 \si{\mega\pascal}
\end{align*}
where $\sigma$ is the yield stress.

The optimized ratio between PLA and PP is therefore 
$
A_\text{PP} / A_\text{PLA} = \sigma_\text{PLA} / \sigma_\text{PP}  \approx 4.3
$

If we choose the width of the PLA beams to be two standard lines wide, \SI{0.7}{\milli\meter}, then the PP beams should be \SI{3.0}{\milli\meter}.




The height of the beams is relatively irrelevant to the optimization.
Only when the beams are a lot higher than the layer thickness does a new failure mode come into play where the beams in the orthogonal directions shear off each other.
Given that the ultimate shear stress is approximately half the ultimate tensile stress,
this failure mode plays a role when the contact area between the beams is half the cross sectional area of a beam.
However, the width of the beams is in the order of twice the nozzle size, whereas the layer height is in the order of a quarter of the nozzle size.
If we keep the beams as high as two layers the shearing failure mode will not take effect.