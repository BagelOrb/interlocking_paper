\section{Related Work}

\subsection{Interlocking}

\begin{itemize}
	\item friction
	\item dovetail
	\item interlocking
\end{itemize}

\subsection{Microstructures}


\subsection{Adhesion}
\citeauthor{tamburrino19} investigate the adhesion between multi-material prints and propose a simple friction based interlocking structure, which consists of straight extrusions of the one material around the other~\cite{tamburrino19}.




\iffalse
% This paragraph explains what toolpath generation is, and the context where toolpath generation is positioned in
%Toolpath generation for deposition based additive manufacturing consists in generating geometric paths which the nozzle follow during printing.
%Extrusion toolpaths are generated within the planar contour of a layer at the intersection of a horizontal plane and a 3D solid object.
Toolpath generation consists in generating a path in the a planar contour, representing the intersection of a plane and a 3D solid object.
The nozzle is then instructed to move along the path while extruding material.
Sites along the toolpath\revise{}{s} are assigned several properties such as movement speed, but for this paper we will focus on the assigned width of the extruded bead.
Toolpath generation is an integral part of process planning for 3D printing.
For an overview of the processing pipeline, we refer to the survey by \citeauthor{Livesu2017CGF}~\cite{Livesu2017CGF}.
For reducing printing time and material cost, sparse infill structures such as triangular and hexagonal patterns have been used to approximate the interior of 3D shapes.
In this paper, we focus on \revise{generating toolpath that seamlessly fills the entire 2D contour.}{the generation of toolpaths for dense regions, such as the boundary shell of 3D shapes.}
This is sometimes called `dense infill'~\cite{Livesu2017CGF}.

% criteria for toolpath generation
The toolpath has a direct influence on the printing time, material cost, and mechanical properties of the printed object~\cite{N.Turner2014,ahn2002anisotropic}.
FDM calls for toolpaths with several desirable properties.
First, the extrusion path should be as continuous as possible.
A discontinuous path requires to stop and restart material extrusion.
For certain materials\revise{}{,} such as TPU, this could lead to printing defects or even print failure~\cite{KUIPERS2019CAD}.
Second, the toolpath is preferred to be smooth.
Sharp turns require to reduce the movement speed of the nozzle, and so this prolongs the printing process.
Third, the extruded path should cover \revise{a }{}the region of the contour without \revise{gaps}{underfilling}.
Such underfill negatively influences the mechanical performance of the parts.
Fourth, the extrusion paths should not overlap with one another.
Such overfill causes a pressure build up in the mechanical system, which leads to overextrusion in \revise{further}{later} paths and in extreme cases cause print failure~\cite{KUIPERS2019CAD}.
An analysis of under- and overfill from a vertical cross-section was presented in~\cite{Han2002JMSE}. 
Our method is primarily concerned with minimizing under- and overfill within horizontal cross-sections.


% basic toolpath strategies and their variations
Two basic strategies for dense toolpath generation are \revise{}{the }direction-parallel\revise{}{ strategy} and \revise{}{the }contour-parallel\revise{}{ strategy}.
Direction-parallel (or zig-zag) toolpaths fill an arbitrarily shaped contour with a set of parallel, equally spaced line segments.
These parallel segments are linked together at one of their extremities, to avoid discontinuous extrusion.
Contour-parallel toolpaths typically consists of a set of equally spaced offsets from the slice boundary outline polygons.
\citeauthor{steuben2016implicit} presented a method for generating sparse infill toolpaths based on the isocontours of surface plots of some variable generated on each 2D contour~\cite{steuben2016implicit}.
In order to increase the continuity of contour-parallel toolpaths, a strategy to connect dense toolpaths into spirals was introduced by \citeauthor{Zhao2016}~\cite{Zhao2016} and later extended to also connect a mixture of dense and sparse toolpaths together~\cite{KUIPERS2019CAD}.
\citeauthor{Jin2017RCIM} discusses several approaches for connecting direction-parallel and contour-parallel toolpaths into continuous paths~\cite{Jin2017RCIM}.
Spiral toolpaths have also been applied to (CNC) machining~\cite{Held2009,Huang2017}.
One of the problems with contour-parallel toolpath\revise{}{s} is that it tends to leave gaps between the toolpaths (see \cref{intro_wedge_uniform}).
This is due to the fact that the diameter of the part is not exact multiple of the (constant) deposition width in those regions.
To avoid problems with such gaps, hybrid approaches that combine direction and contour-parallel are often used~\cite{Mcmains2000DETC,Jin2013adaptive}.
Close to the slice boundary, there are several contour-parallel curves, while \revise{in }{}the interior is filled using a zig-zag pattern.
For complex shapes, the entire cross-section could be decomposed into a set of patches, and for each of them the basic strategies can be applied~\cite{Ding2014,Jin2017RCIM}.
Alternative toolpath patterns, seen also in CNC machining, include space-filling curves~\cite{Cox1994CAD,Griffiths1994,Shaikh2016}.

% the idea, and (some of) the most related work
Reducing under- and over-filling can be accurately achieved by making use \revise{over}{of} adaptive \revise{depisition}{deposition} width.
Adaptive width can be used to locally match the nonuniform space between adjacent paths, and thus to ensure a better filling of the area.
%Material extrusion with a varying width can be realized by changing the extrusion rate or the nozzle travel velocity~\cite{Ertay2018,Kuipers2018}.
\citeauthor{kao1998optimal} propose smooth adaptive toolpaths based on the medial axis skeleton of the boundary contour~\cite{kao1998optimal}.
Their approach handles simple geometry where there are no branches in the medial axis.
An extension was proposed by \citeauthor{Ding2016a} to handle complex shapes~\cite{Ding2016a}.
However, this extension inherits a problem in the original method:
from any point in the skeleton to the boundary, the number of toolpaths is constant.
\revise{%sdg
Since the distance may vary considerably for shapes with both large and small features, this strategy leads to a toolpath with widths that may differ by a factor beyond $3$, which is impossible for some FDM systems.
}{%f
Depending on the size of small and large features in the layer outlines, this strategy can require a range of bead widths beyond the capabilities of the manufacturing system.
}%sagd
\citeauthor{Jin2017JMS} proposed a strategy of adding \revise{a }{}toolpath\revise{}{s} with varying width along the center edges of the skeleton, while leaving other paths unchanged~\cite{Jin2017JMS}.
The resulting \revise{variation of width in the center is still up to a factor of $2$}{beads have widths within the wide range of $[0.25w,1.8w]$} (see \cref{intro_wedge_centered}).
In this paper we propose a novel scheme to distribute the width alterations throughout a region around the center, and thus limit the occurrence of extreme \revise{deviations}{variation} in width (see \cref{intro_wedge_distributed}).

% the context where our method can be positioned, and some related problems in that context
Under- and over-filling issues have a large proportional impact for thin geometric features.
\citeauthor{Jin2017a} proposed a sparse wavy path pattern for thin-walled parts~\cite{Jin2017a}.
Besides under- and over-filling, there are a few other robustness issues in toolpath generation for thin geometric features.
\citeauthor{Moesen2011} proposed a method to \revise{make beam compensation more reliable for}{reliably manufacture} thin geometric features\revise{}{ using laser-based additive manufacturing techniques}~\cite{Moesen2011}.
>> achange!!
\citeauthor{Behandish2019a} presented a method to characterize local- topological discrepancies due to material under- and over-deposition, and used this information to modify the as-manufactured outcomes~\cite{Behandish2019a}. 

\fi

% \subsection{Toolpath strategies}
% Simple Zig-Zag toolpathing strategy~\cite{mcmains2000layered}.
% Patchwise Zig-Zag toolpathing strategy~\cite{Ding2014}.

% Combining concentric toolpaths into continuous extrusion spirals~\cite{Held2009}.

% Fractal based toolpaths~\cite{Griffiths1994, mandal2016}.

% \subsection{Space filling toolpaths}
% \todo{find other literature which tries to minimize underfilling and overfilling which is not based on a skeletonization.}


% \subsection{Medial axis based toolpaths}
% Adjusted Medial Axis Transform (MAT) structure for only printing the outer wall, rest area to be filled using normal infill~\cite{Moesen2011}.
% \Cref{moessen}
% Problem: small grey areas which are too small for the second wall lines.

% Figuring out underfilling and overfilling arteas in concentric fill and using single squigly lines to prevent overfilling~\cite{Jin2017JMS}.
% \Cref{jin}

% Using variable width lines to fit a precise amount of lines using the MAT.
%~\cite{Ding2016a} apply the method from~\cite{kao1998optimal}.
% \Cref{ding}

% \begin{figure}
% \begin{subfigure}{0.45\columnwidth}
% \includegraphics[width=\columnwidth]{sources-related-work-moessen.jpg}
% \caption{\citeauthor{Moesen2011}}
% \label{moessen}
% \end{subfigure}
% \begin{subfigure}{0.45\columnwidth}
% \includegraphics[width=\columnwidth]{sources-related-work-jin.jpg}
% \caption{\citeauthor{Jin2017JMS}}
% \label{jin}
% \end{subfigure}
% \end{figure}

% \begin{figure}
% \centering
% \includegraphics[width=\columnwidth]{sources-related-work-ding.jpg}
% \includegraphics[width=.7\columnwidth]{sources-related-work-kao.jpg}
% \caption{Path planning strategies proposed by Kao (bottom) and employed in FDM by Ding et al (top).}
% \label{ding}
% \end{figure}

% \subsection{Variable extrusion width}

% Changing extrusion rate and temperature to match a given velocity~\cite{Ertay2018}.

% Changing velocity in order to change extrusion rate.
% Shortly discussed in~\cite{Kuipers2018}.
% \todo{find another reference}