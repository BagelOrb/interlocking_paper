% !TeX spellcheck = en_GB 
\section{Interlocking}
\subsection{Interlocking upper strength bound}
at any cross section the total force must be present in both materials

\begin{align*}
	\sigma &= \frac{F}{A_a + A_b} \\
	&= \sigmafail{a} \frac{A_a}{A_a + A_b} \\
	&= \sigmafail{b} \frac{A_b}{A_a + A_b} \\
\end{align*}

\begin{align}\label{eq:general_uper_bound}
	\sigma^* &= \sigmafail{b} \left(\frac{ \nicefrac{\sigmafail{a}}{\sigmafail{b}} }{ \nicefrac{\sigmafail{a}}{\sigmafail{b}} + 1 } \right) \nonumber \\
	&= \sigmafail{b} \left(\frac{ 1 }{ 1 + \nicefrac{\sigmafail{b}}{\sigmafail{a}} } \right)  \nonumber \\
	&= \frac{ 1 }{ \nicefrac{1}{\sigmafail{b}} + \nicefrac{1}{\sigmafail{a}} } 
\end{align}

\subsection{Genus interlocking}
What is genus interlocking?

Less dependency on friction and deformability (poissons ratio?)








\section{Interlaced Genus Interlocking Structure}
Basic structure; Connectivity graph

Stacked vs single

straight vs diagonal

optimize for tensile


\subsection{Straight}

stress formulae

bound constraints

problem formulation

scale invariance, neccesarily active constraints

inversion

response surfaces; active constraints

\subsubsection{broken optimum}


changed problem definition




\subsection{Diagonal}
symmetry; design variables

Manufacturability;Rounded edges with radius $r$

formula for width
\begin{align}
	d &= 2Mw / \sqrt{4M^2-w^2} \\
	M &= L - 2r \\
	w &= \wa + \wb
\end{align}

area of Z shear
\begin{align}
	A_z &= \frac12 \pi r^2 + r (\wa-2r) \frac{d}{w} + d M \left(\frac{\wa}{w}\right)^2
\end{align}


difficult to model; homogeneous stress distribution assumption violated; deformation dependent

assume homogenous tensile stress as if beam was straight?
