% !TeX spellcheck = en_GB 
\section{Related Work}


\subsection{Microstructures}
% multi-material microstructures
Microstructures such as beam lattices, triply periodic surfaces and ordered dithering have widely been studied for their mechanical properties\cite{Cadman2013,Zhang2018a,tamburrino2018}.
Single material microstructures can constitute auxetic materials, or light weight structures with tailored material properties, through fine adjustments of the microstructure geometry.
Such microstructures are often manufactured using stereolithography and polyjetting, because these processes have very few manufacturing constraints.
Microstructures inherently have a high genus topology, which makes them good candidates for interlocking\cite{freund2019determination}.
However, optimizing microstructures for adhesions between incompatible materials while adhering to FDM manufacturing constraints has been studied scarcely.



\subsection{Adhesion}

Important factors for adhesion between polymers are entanglement and dissipation\cite{abbott2015adhesion}.
The adhesion by which two bodies of different FDM printed material stick together can be influences by a wide spectrum of pre-treatment methods, process parameters and material properties\cite{freund2019determination}.
Increasing surface roughness might improve the adhesion between materials\cite{huttenbach1991interface,gent1990model}, but this supposed benefit is contested\cite{abbott2015adhesion}.
For FDM one could try mixing the materials by overlapping their toolpaths to increase adhesion or create simple straight protrusions in order to increase the friction between the two materials\cite{tamburrino19}.
However, if such protrusions bulge outward the adhesions doesn't merely increase through friction, but also because it would constitute a dovetail type of interlocking structure.

% Cura 4.8 allows for generating sandwich structures where the layers are alternated between the two materials.


\subsection{Interlocking in FDM}

%\subsubsection{Dovetail interlocking}
Literature on interlocking patterns for adhesion between incompatible materials in FDM 3D printing has focussed on extruded 2D interlocking dovetail type of designs,
such as jigsaw shapes ($\Omega$)\cite{malik2017}, trapezoidal sutures ($\sqcap$)\cite{Li2013}, T-shapes\cite{Ribeiro2019,mustafa2021development}, star shapes ($\bigstar$)\cite{Wang2021} in the horizontal direction
as well as in the vertical direction across several layers\cite{debora2020}.
Topology optimization can generate complex 2D dovetail interlocking shapes, which fit to the specifics of the design locally\cite{aharoni2021}.
Such interlocking designs are extrusions of 2D shapes and therefore they allow for translation along the extrusion axis, so these type of interlocking designs don't lock all axes.

The jigsaw idea can be expanded into a 3D interlocking structure by protruding not only sideways, but also vertically, resulting in a shape which looks like a tree ($\clubsuit$)\cite{gouker2006manufacturing}.
Similarly the T-shaped interlocking design with horizontal bars can be expanded with bars in the vertical directions.
However, such designs violate the semi-continuous extrusion requirement because the layers above and below the base of the T contain separated islands of one material, 
which are difficult to print when the two materials don't adhere to each other.

%\subsubsection{Topological interlocking}
These issues can be addressed by generating a repeating microstructure where the interlocks are connected together to form a high genus topology which locks all degrees of freedom.
Simple straight I shaped extrusions can be linked together by cross beams in order to form a topologically interlocking structure\cite{Rossing2020}.
In case of the application where the second material is overmolded silicone the extrusion continuity constraint doesn't apply,
but in cases where the second material is also 3D printed, the space enclosed by the cross beams and extruded walls forms an island which introduces an extrusion discontinuity.
A different microstructure is required if the extrusion continuity constraint is to be met for both materials.

%%  Single material adhesion structures
%The mechanical properties of single material 3D prints can also be improved by techniques similar to interlocking.
%By injecting material which span several layers the inter-layer bonding can be improved\cite{Duty2019,Kazmer2020}.
%Weaving extruded strands together across several layers is another technique to improve layer adhesion\cite{yao20213d}.


