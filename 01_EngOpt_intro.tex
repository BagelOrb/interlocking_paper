\begin{table}
	\caption{Statement of Contributions}
	\begin{tabular}{lll}
		& Tim & Nadine \\
		Introduction &  \checkmark & \\
		Problem formulation & Straight & Diagonal \\
		Problem investigation & Straight & Diagonal \\
		Optimization algorithm & \checkmark & \checkmark \\
		Optimization section & \checkmark &  \\
		Results &  & \checkmark \\
		Discussion & & \checkmark \\
		Conclusion & \checkmark & \\
	\end{tabular}
\end{table}



\section{Introduction}
\textit{This project is centered around the PhD research of Tim Kuipers. 
	Therefore we have done slightly more work than necessary to obtain the right model and optimize the structure.}

\medskip

If one wishes to print a part consisting of two incompatible materials, an interlocking geometry is needed at the interface to make both materials mechanically adhere to each other. 
This can be achieved with structure that consists of parallel beams of alternating materials that interlock. 
Two materials were considered, a hard one (shown in green) and a soft one (in cyan), which are placed such that the interface is perpendicular to the printed layers, see \cref{fig:failure_modes}.

The structure needs to be optimized for the highest ultimate strength of the interlocking micro-structure to maximize the force that a given area can withstand. 
It will be loaded in tension which is applied orthogonally to the interface. 
None of the components is allowed to break or yield at the highest applied force.

In addition, the structure shall be as small as possible to prevent that it affects the full part design, which is captured in both the constraints and the objective functions.
The design space of both geometries is limited by the following manufacturing constraints.
\begin{itemize}
	\setlength\itemsep{0mm}
	\item The layer height is a discrete multiple of the layer thickness $h_{min}$.
	\item Any geometry height is at least $2 h_{min}$ to prevent oscillations during manufacturing.
	\item The total height of the micro-structure shall be less than 12 layers, therefore $h_{max}=12 h_{min}$.
	\item Any geometry length is least $2w_{min}$ to prevent frequent retractions.
	\item The total length of the micro-structure shall be less than 12 layers, therefore $L_{max} = 12 w_{min}$.
	\item Any geometry width is larger than the minimum manufacturable thread width $w_{min}$, which is determined by the nozzle size.
	\item The total width of the micro-structure shall be less than $12w_{min}$.
\end{itemize}

The nozzle size of the printer used is \SI{0.4}{\milli\meter}, $w_{min}$ is estimated to be slightly lower with a value of $\SI{0.3}{\milli\meter}$.
The layer height is set at $h_{min}=\SI{0.1}{\milli\meter}$.

Two orientations of the interlocking pattern are investigated:
\begin{description}
	\setlength\itemsep{0em}
	\item[Straight] Even beams (`cross beams') are aligned parallel and odd beams (`fingers') are placed perpendicular to the interlocking interface, see \cref{fig:failure_modes}.
	\item[Diagonal] Even beams and odd beams (both `fingers') are oriented at an angle with respect to the interlocking interface, see \cref{fig:diagonal_model}.
\end{description}

The materials that were considered are material $a$: Ultimaker Green Tough PLA and material $b$: Ultimaker Transparent PP. 
TPU might be considered as well in further analysis. 
The material properties are given in \cref{tab:mat_props}.
Material $a$ is taken as the stronger material: $\sigma^a_\text{yield} > \sigma^b_\text{yield} $.
Note that 3D printed structures are anisotropic.
These material properties are taken for load cases parallel to layer width since the behavior of a structure is determined mostly by the walls of a print when loaded in tension. 
The current model only involves $\sigma_\text{yield}$ and $\sigma_\text{z,yield}$. 
Since TPU does not yield, the ultimate strength will be used instead.


\begin{table}
	\centering
	\caption{Material properties according to Ultimaker technical data sheets. \todo{Starred* data will have to be rechecked.}}
	\label{tab:mat_props}
	\begin{tabular}{lrrrl}
		& PLA & PP & TPU & \\
		$E$ & 2797 & 302 & 67 & \si{\mega\pascal} \\
		$E_z$ & 2696 & 262 & 56 & \si{\mega\pascal} \\
		$\sigma_\text{yield}$ & 47 & 10.5 & - & \si{\mega\pascal} \\
		$\sigma_\text{z,yield}$ & 33 & 9.0* & - & \si{\mega\pascal} \\
		$\epsilon_\text{yield}$ & 3.5 & 29 & - & \si{\percent} \\
		$\epsilon_\text{z,yield}$ & 2.6 & 25* & - & \si{\percent} \\
		$\sigma_\text{break}$ & 31 & 9.4 & 38 & \si{\mega\pascal} \\
		$\sigma_\text{z,break}$ & 32 & 7.8* & 6.4 & \si{\mega\pascal} \\
		$\epsilon_\text{break}$ & 8.2 & 70* & $>700$ & \si{\percent} \\
		$\epsilon_\text{z,break}$ & 3.1 & 50* & 82 & \si{\percent} \\
	\end{tabular}
\end{table}




