\section{Optimization Approach}
The SQP algorithm was chosen to optimize problem. 
Given that the optimization problem is constrained, none of the unconstrained optimization methods applies.
Also, given that the objective function and all of the stress values are monotonic, the surface is not expected to show high irregularies or discrete jumps. Therefore random methods like random jumping, simulated annealing would be inefficient and these are neither repeatable, which is desired when comparing two cases. The same holds for the genetic algorithm which is computationally expensive for a large population and number of iterations.
Since the governing equations are available, the first and second order derivatives can be utilized to reach the optimum with less computational power. Therefore, neither Nelder-Mead Simplex nor Powell’s conjugate directions were considered.\\

First order methods, like the conjugate gradient method can also be used, however it has a slow convergence for non quadratic functions which appear in the design problem. Considering second order methods, SQP is the best general purpose method for differentiable constrained problems. It can handle inquality constraints with an active set strategy and it does not require a feasible starting point. Since the objective function and constraints are differentiable and SQP is expected to converge fast towards the optimum, SQP will be used to evaluated the optimization problem.
 



\subsection{Symbolic implementation}

\newcommand{\xvec}{\mathbf{x}}
\newcommand{\hvec}{\mathbf{h}}
\newcommand{\gvec}{\mathbf{g}}
\newcommand{\Wmat}{\mathbf{W}}
\newcommand{\Amat}{\mathbf{A}}
\newcommand{\lamvec}{\bm{\lambda}}

Our objective and constraint formulae are differentiable,
but deriving the derivative of the more complex constraints may prove to be laborsome and error-prone.
We therefore use the Matlab builtin functionality \verb|diff| to derive the derivatives of the symbolic formulae.
Performing the differentiation like this is computation intensive, but since it can be performed at the start of the program we can save on precious computation time inside the main loop.
The symbolic expressions are then converted to standard functions using \verb|matlabFunction|, so that they can be evaluated efficiently inside the loop.

The $\Wmat$ and $\Amat$ matrices can then be constructed on the fly using the formulae
$\Wmat = \diffp[2]{f}{\xvec} + \lamvec^\intercal \diffp[2]{\hvec}{\xvec}$
and
$\Amat = \diff{\hvec}{\xvec}$.
Note that $\diffp[2]{\hvec}{\xvec}$ is a precomputed 3D matrix.
We can then use \verb|quadprog| to solve $\min_{\Delta\xvec} \nicefrac12 \Delta\xvec^\intercal \Wmat\Delta\xvec + \nabla f^\intercal \Delta\xvec$ s.t. $\Amat\Delta\xvec+\hvec=\mathbf{0}$.

\subsection{Active set}
It was chosen to implement an active set strategy, because using slack variables or using penalty or barier functions introduces inaccuracies in the final result.
Here $\hvec$ is an active subset of all constraints $\gvec$, which is determined in each iteration based on the current $\xvec_k$.
One simple approach would be to set all constraints active which are violated by the current $\xvec_k$.
In order to be lenient to numerical errors we set a constraint $g_i$ as active when $g_i > -10^{-4}$.
This way successive iterations along a constraint boundary won't oscillate the active set.

However, such an approach can lead to problems when considering points near or beyond the global optimum.
When the active set contains more constraints than design variables, the quadratic program is unsolvable.
In an N-dimensional space only N hypersurfaces generally intersect in a point.
We feel that SQP is not inherently equipped to deal with an active set larger than the design space.

One naive approach to alleviate the issue is to choose the subset of the most violated constraints.
However, this will not work, since the optimization will only push $\xvec_k$ further from the ignored constraints if that is in favor of the objective function.
We did not develop any strategy to deal with this problem;
the program terminates prematurely with an error message in such a case.

\subsection{Lagrange multipliers}
In order to compute the lambda multipliers of the outer problem we solve the system of linear equations given by
$\Amat_k =  - \Wmat_k  \Delta\xvec - \left[\nabla f\right]_k
$

Because the active set can change each iteration, the lambda values of constraints which were unused in the previous iteration might have to be revived.
We keep track of a set of Lagrange multipliers out of which the active subset is used each iteration.
We store the lambda values of the previous iteration in the full set, update the active set and then retrieve the new set of multipliers.

However, because the lambda values signify the relative importance of the constraints w.r.t. each other, the lambda values of one set cannot be reliably compared to another active set from the previous iteration.
We therefore set $\lamvec = \mathbf{1}$ whenever the active set changes.

\subsection{Instability prevention}
In some cases the optimization can oscillate between two constraints.
It can happen that the optimizatoin cycles between points on either side of two constraint surfaces.
Each time the optimization is in a point $\xvec_k$ which violates some constraing $g_a$,
it doesn't violate the other constraint $g_b$, which is then ignored;
the next iteration is then the other way around and thus a cycle is created.
However, this type of cycle is a cycle in the active set, not in $\xvec$.
We therefore invented a strategy to see if there is a cycle of 2 in the active sets of the past 6 iterations.
If there is, we force the active set to be the union of the 2 active sets in the cycle.
The forced active set is alleviated after 6 iterations, so that new constraints can come into view of the optimization.

Another problem which often occurs in one of the first iterations is that the update $\Delta\xvec$ becomes too large.
The optimization can then quickly escalate.
We therefore implemented a simple type of move limits: the magnitude of $\Delta\xvec$ was constrained to a maximum Euclidean length of 1.




\subsection{Stopping criteria}
The main loop of the optimization method is concluded when either of several criteria is met.
We break the loop when the improvement in one iteration is too small: $\left| \Delta\xvec \right| < \bm{1} \cdot 10^{-10}$.
In order to stop even earlier than that we also check the KKT conditions every iteration.
We can use \verb|linsolve| to get a full set of Lagrange multipliers to satisfy the optimality criterion
and verify whether the feasibility and complementarity constraints are satisfied.
However, using the approximate $\lamvec$ seems to give the same results.


In other cases the main loop is terminated because the optimization failed.
In order to prevent an infinite loop, we set the maximum number of iterations to 500.
Also, in case there are more active constraints than design variables,
the \verb|quadprog| algorithm would fail with an error about Non-convexity,
because of problems with the active set as noted above.
We also implemented a typical cycle detection, based on the values of $f$ and $\xvec$.
If the difference between a point $\xvec_k$ and $\xvec_j$ is smaller than $10^{-4}$ in all dimensions of $\xvec$ and $f$ then execution is halted.
In order to prevent the failure, our program terminates prematurely with an error message.

