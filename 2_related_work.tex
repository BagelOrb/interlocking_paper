% !TeX spellcheck = en_US 
\section{Related Work}\label{interlocking:sec:related_work}

\subsection{Multi-material additive manufacturing}
Multi-material additive manufacturing has unlocked a plethora of applications, by making use of functionally graded material properties, tailored composite materials or multi-material designs.
Several review papers cover a wide range of techniques on these topics.\cite{Vaezi2013,Rafiee2020}
Using multiple materials to create coloured surface imagery is commonly performed using MultiJet technologies, but can alternatively be performed using inkjet techniques\cite{sachs1994three} or even DLP resin printers \cite{Zhou2011Development}.
Such techniques can even be extended to deal with translucency\cite{Brunton2018} and gloss\cite{elkhuizen2019gloss}.
Several colour printing techniques have also been proposed for \revise{FDM}{MEX}.\cite{reiner2014dual,Song2019,Kuipers2018}

Besides visual attributes, multi-material systems can be used to create parts with graded material properties, by generating composite structures with varying densities of soft and hard materials.\cite{Cho2003851}
Fine-tuning the small-scale geometry in which such materials are deposited can give a more sophisticated control over the induced material properties,\cite{Leung2019,mirzaali2020}
and adjusting the small-scale geometry throughout the product on a \revise{mesa}{meso} scale can increase the performance of the product even more.\cite{Zhu2017}

Some \revise{FDM}{MEX} systems for multi-material extrusion have been proposed which operate by extruding multiple materials out of a single nozzle, e.g. by routing multiple filaments into a single mixing nozzle\cite{diamondhotend} or by creating a single strand of multi-material filament\cite{Takahashi2020,Mosaic2015},
but such systems can exhibit hardware issues when the different materials require vastly different processing parameters.
Therefore the more upscale multi-material extrusion systems use a separate nozzle for each material.\cite{UltimakerS5}

% \cite{DeBacker2018,Yu2020} fibre reinforced \revise{FDM}{MEX}


\subsection{Microstructures}
% multi-material microstructures
Microstructures such as beam lattices, triply periodic surfaces and ordered dithering have widely been studied for their mechanical properties.
Several review papers provide a comprehensive overview of such microstructures.\cite{Cadman2013,Zhang2018a,tamburrino2018}
Single material microstructures can constitute auxetic materials, % TODO citation needed
or light weight structures with tailored material properties, through fine adjustments of the microstructure geometry. % TODO citation needed
%Such microstructures are often manufactured using stereolithography and polyjetting, because these processes have very few manufacturing constraints.
Multi-material microstructures inherently exhibit topological interlocking, which makes them good candidates for interlocking\cite{freund2019determination}.
However, optimizing microstructures for adhesions between incompatible materials while adhering to \revise{FDM}{MEX} manufacturing constraints has been studied scarcely.

% aremu2017voxelbased 


\subsection{Adhesion}

Important factors for adhesion between polymers are entanglement and dissipation\cite{abbott2015adhesion}.
The adhesion by which two bodies of different \revise{FDM}{MEX} printed material stick together can be influences by a wide spectrum of pre-treatment methods, process parameters and material properties\cite{freund2019determination}.
Increasing surface roughness might improve the adhesion between materials\cite{huttenbach1991interface,gent1990model}, but this supposed benefit is contested\cite{abbott2015adhesion}.
For \revise{FDM}{MEX} one could try mixing the materials by overlapping their toolpaths to increase adhesion or create simple straight protrusions in order to increase the friction between the two materials\cite{tamburrino19}.
However, if such protrusions bulge outward the adhesion does not merely increase because of the increased friction, but also because it would constitute a dovetail type of interlocking structure.

% Cura 4.8 allows for generating sandwich structures where the layers are alternated between the two materials.


\subsection{Interlocking in \revise{FDM}{MEX}}

%\subsubsection{Dovetail interlocking}
Literature on interlocking patterns for adhesion between incompatible materials in \revise{FDM}{MEX} 3D printing has focussed on extruded 2D interlocking dovetail type of designs,
such as jigsaw shapes \cite{malik2017}, trapezoidal sutures \cite{Li2013}, T-shapes\cite{Ribeiro2019,mustafa2021development} and star shapes\cite{Wang2021} in the horizontal direction
as well as in the vertical direction across several layers\cite{debora2020}.
Topology optimization can generate complex 2D dovetail interlocking shapes, which fit to the specifics of the design locally\cite{aharoni2021}.
\revise{Such}{In as much as such} interlocking designs are \revise{}{straight }extrusions of 2D shapes \revise{and therefore }{}they allow for translation along the extrusion axis, so these type of interlocking designs do not lock all axes.\footnote{\revise{}{The addition of a layer of material on top and below the extrusions is a simple ad hoc method to lock the translation, but such an addition is not an inherent element of such lattices.}}

The jigsaw idea can be expanded into a 3D interlocking structure by protruding not only sideways, but also vertically, resulting in a shape resembling a tree\cite{gouker2006manufacturing}.
Similarly the T-shaped interlocking design with horizontal bars could be expanded with bars in the vertical directions.
However, such designs violate the semi-continuous extrusion requirement because the layers above and below the base of the T would contain separated islands of one material, 
which are difficult to print accurately when the two materials do not adhere to each other.

%\subsubsection{Topological interlocking}
These issues can be addressed by generating a repeating microstructure where the interlocks are connected together;
simple straight I shaped extrusions can be linked together by cross beams in order to form a topologically interlocking structure resembling a ladder\cite{Rossing2020}.
Such a topologically interlocking structure satisfies the continuity constraint only for a single material and is therefore interesting for applications where the one material is 3D printed, while the other is overmolded silicone.
However, a different microstructure is required if the extrusion continuity constraint is to be met by both materials.

%%  Single material adhesion structures
%The mechanical properties of single material 3D prints can also be improved by techniques similar to interlocking.
%By injecting material which span several layers the inter-layer bonding can be improved\cite{Duty2019,Kazmer2020}.
%Weaving extruded strands together across several layers is another technique to improve layer adhesion\cite{yao20213d}.


