\section{Introduction}


A common type of interlocking can be found in jigsaw puzzles.
The pieces interlock and stay connected because of the material stiffness.
The interlock can be broken in plane by deformation of the pieces.
Out of plane the pieces can be disassembled easily.

A better type of interlocking can be found in a chain.
The only way to break a chain is by breaking one of the links.
They cannot be broken by deformation alone, because links of a chain have a topology of genus 1.
We therefore consider high genus interlocking structures.

For the application of dual material FDM 3D printing we propose a high genus interlocking structure where holes in the one material are filled with the other material and vice versa.
Introducing discontinuities in the extrusion process leads to defects, which influence the accuracy and the mechanical properties of the resulting part.
Moreover, depositing disconnected islands of one material on top of a chemically incompatible material cannot reliably be performed since the island isn't fastened to anything.
We therefore have to generate interlocking geometry for which the layers consist of continuously connected areas.

Intuitively it seems impossible to generate high genus interlocking geometry while enforcing continuous extrusion for both materials;
if the one material leaves a hole in a layer then filling that hole with the other material will cause it to be disconnected from the other regions of the second material.
The trick is to align all holes horizontally, so that they never cross a plane at which a layer is sliced.



Our contributions are as follows:
\begin{itemize}
	\item A high genus interlocking structure for continuous extrusion
	\item A similar structure optimized for bending (?)
	\item A similar structure optimized for tensile pulling (?)
\end{itemize}

Are we gonna model and/or test the pattern in the Z direction?
Z forces should be taken into account to show that our method also extends to horizontal interfaces.
The Z direction is a major contribution compared to existing literature.

Are we gonna make some algorithm which aligns the pattern with the interface similar to Lars' paper?

Is simulation validation or part of the method?
How does simulation relate to the analytical model?

The analytical model is more general, but less well validated.
The analytical model can influence the starting point of the simulation optimization.
The simulation can verify the analytical model.


What applications are we going to focus on?
Maybe do an application where we do in-situ optimization?

Something with living hinges.









