% !TeX spellcheck = en_GB 
\subsection{Simulation}
In order to simulate interlocking structures with a range of design parameters we automatically generate an INP file in Abaqus CAE (2020) using a script.
Solving the INP file gives us the force-displacement graph, from which we can determine the ultimate tensile strength.
In order to simulate accurately we used the stress-strain curves from tensile tests on the base materials printed flat on the build plate as the plasticity in tabular form.
The simulations were performed in the Abaqus/Explicit solver where Dynamic, Explicit procedure step was used with the mass scaling factor of $10^7$,
using geometric nonlinearity and general contact (explicit) to disregard friction for simplicity.

The repeating nature of the interlocking patterns was captured by modelling half of the unit cell and apply symmetry constraints to the sides, top and bottom.
The model was meshed using C3D8R hexahedral elements of $\pm\SI{75}{\micro\meter}$.

A grid search was used to measure the influence on the ultimate strength along each of the design variables $\wb (,\va, \hc)$, along with the total length $L$.
The search space was therefore 4D and 2D for the straight and diagonal design respectively.
In order to estimate the optimum we fit a smooth response surface to these data points using a radial basis function (RBF) network\cite{Dinh2002},
where a smoothness of $\lambda=1$ produces satisfactory results.


\subsubsection{Straight}
In order to prevent stress concentrations and adhere to manufacturing accuracy, the vertical edges of the straight design were rounded with $r=\SI{0.15}{\milli\meter}$;
see \cref{fig:sim_straight_model}.
We performed two rounds of hypersurface fitting on grid search; the second round was in a zoomed in region and with elements of $\pm\SI{50}{\micro\meter}$.

Newton's method was used to determine the optimum, starting from the best sampled point.
This step only considered the dimensions $\wb$ and $\va$, because $\lmax$ is given and $\hf$ has to be an integer multiple of $\hmin$.
It's unlikely the optimum of the fitted hypersurface would be on a different integer multiple of $\hmin$.
The resulting hypersurfaces are visualized in \cref{fig:simulation_results_straight}.
The obtained optima are shown in \cref{tab:sim_straight_optima}.

We compare the results of round 1 to our analytical model by adjusting the analytical model to capture the inaccurate Z strength used in the simulations:
$\sigmafailz{m} := \sigmafail{m}$.
We then observe that our analytical model on average predicts only \SI{7.8}{\percent} higher ultimate strength values than then the FEM simulations, with a standard deviation of \SI{16.2}{\percent}.

\begin{figure}
	\centering
	\begin{subfigure}[B]{.45\columnwidth}
		\centering
		\includegraphics[width=\columnwidth]{sources/simulation/mesh-pp.jpg}
		\caption{PP}
	\end{subfigure}
	\begin{subfigure}[B]{.45\columnwidth}
		\centering
		\includegraphics[width=\columnwidth]{sources/simulation/mesh-pla.jpg}
		\caption{TPLA}
	\end{subfigure}
	\caption{Example simulation mesh of the straight design.}
	\label{fig:sim_straight_model}
\end{figure}



\begin{figure*}
	\centering
	\begin{subfigure}[B]{.49\columnwidth}
		\centering
		\includegraphics{sources/simulation/r12-lmax3.6.pdf}
		\caption{$\lmax=\SI{3.6}{\milli\meter}; \hf=\SI{0.8}{\milli\meter}$}
	\end{subfigure}
	\begin{subfigure}[B]{.49\columnwidth}
		\centering
		\includegraphics{sources/simulation/r12-lmax3.0.pdf}
		\caption{$\lmax=\SI{3.0}{\milli\meter}; \hf=\SI{0.8}{\milli\meter}$}
	\end{subfigure}
	\begin{subfigure}[B]{.49\columnwidth}
		\centering
		\includegraphics{sources/simulation/r12-lmax2.4.pdf}
		\caption{$\lmax=\SI{2.4}{\milli\meter}; \hf=\SI{0.7}{\milli\meter}$}
	\end{subfigure}
	\begin{subfigure}[B]{.49\columnwidth}
		\centering
		\includegraphics{sources/simulation/r12-lmax1.8.pdf}
		\caption{$\lmax=\SI{1.8}{\milli\meter}; \hf=\SI{0.6}{\milli\meter}$}
	\end{subfigure}
	\caption{2D slices of the 4D simulation results and fitted hypersurface for the straight design. Sampled data points in black, optimum in white.}
	\label{fig:simulation_results_straight}
\end{figure*}

\begin{table}
	\caption{Optimal designs according to the hypersurface fitted to the FEM simulations for the second round of the straight model and for the diagonal model.}
	\label{tab:sim_straight_optima}
	\begin{tabular}{ll|llll}
		&$\lmax$ (\si{\milli\meter})             & 3.6 & 3.0 & 2.4 & 1.8 \\
		\hline
		\multirow{4}{*}{\rotatebox[origin=c]{90}{straight}}
		&$\sigma_\text{max}$ (\si{\mega\pascal}) & \bf 6.11 & \bf 6.03 & \bf 5.81 & \bf 5.53 \\
		&$\hf$ (\si{\milli\meter})               & 0.8 & 0.8 & 0.7 & 0.6 \\
		&$\wb$ (\si{\milli\meter})               & 2.54 & 2.35 & 2.22 & 2.01 \\
		&$\va$ (\si{\milli\meter})               & 2.82 & 2.27 & 1.84 & 1.36 \\
		\hline
		\multirow{2}{*}{\rotatebox[origin=c]{90}{diag}}
		&$\sigma_\text{max}$ (\si{\mega\pascal}) & \bf 6.40 & \bf 6.45 & \bf 5.82 & \bf 4.72 \\
		&$\wb$ (\si{\milli\meter})               & 1.2 & 1.2 & 1.2 & 0.9 \\
		\end
		{tabular}
\iffalse
	\begin{tabular}{l|llllllll}
		Round & 1 & 2 & 1 & 2 & 1 & 2 & 1 & 2 \\
		\hline
		$\lmax$ (\si{\milli\meter}) & 3.6 & 3.6 & 3.0 & 3.0 & 2.4 & 2.4 & 1.8 & 1.8\\
		$\hf$ (\si{\milli\meter}) & 0.8 & 0.8 & 0.8 & 0.8 & 0.7 & 0.7 & 0.6 & 0.6 \\
		$\wb$ (\si{\milli\meter}) & 2.58 & 2.54 & 2.42 & 2.35 & 2.18 & 2.22 & 2.05 & 2.01\\
		$\va$ (\si{\milli\meter}) & 2.67 & 2.82 & 2.23 & 2.27 & 1.78 & 1.84 & 1.35 & 1.36 \\
		$\sigma_\text{max}$ (\si{\mega\pascal}) & 6.17 & 6.11 & 6.12 & 6.03 & 5.89 & 5.81 & 5.59 & 5.53
	\end{tabular}
\fi
\end{table}



\subsubsection{Diagonal}
Modelling the diagonal design in Abaqus can be quite cumbersome, since it doesn't natively support periodic boundary constraints.
Whereas this problem can be overcame in the straight design because it is symmetric,
the diagonal design is rotationally symmetric.
While a symmetry constraint can be used on the top and bottom, the two sides of the design are mirror images of each other, but also flipped vertically.

However, since the height of the beams is relatively low compared to their width we have observed that the stresses and strains are quite similar in the top and bottom.
If we model half of the diagonal cell by cutting it vertically and apply symmetry constraints to the sides,
the induced error is only approximately 10\% compared to simulating an interface consisting of two whole cells.
% small discontinuities



\begin{figure}
	\centering
	\begin{subfigure}[B]{.49\columnwidth}
		\centering
		\includegraphics[width=\columnwidth]{sources/simulation/diagonal_sim_response.jpg}
		\caption{Linear interpolation}
	\end{subfigure}
	\caption{Diagonal results.}
	\label{fig:sim_diagonal_model}
\end{figure}


The results of these simulations are shows in \cref{fig:sim_diagonal_model}.
The predictions from the analytical model can directly be compared to the simulation results, because the constraints on Z shear are not active.
The analytical model predicts 6\% lower ultimate strength values on average with a standard deviation of 15\%.

%optimal designs table








